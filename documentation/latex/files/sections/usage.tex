% Usage Examples - Development mode, examples, libraries, version history
\section{Usage Examples}

\subsection{Development Mode}

Run backend and frontend in development:

\begin{verbatim}
# Backend
cd backend
cargo run -- --dev

# Frontend (separate terminal)
cd frontend
npm install
npm run dev
\end{verbatim}

In development mode, the email verification link is typically logged to the console or terminal instead of sending a real email. This allows manual testing of the full verification flow without configuring an SMTP server.

\subsection{Registration Flow Example}

\begin{lstlisting}[language=Java]
import { registerUser } from '@/services/api';

async function handleRegister(): Promise<void> {
  const response = await registerUser({
    username: 'john_doe',
    email: 'john@example.com',
    password: 'SecureP@ssw0rd123',
  });

  if (response.success) {
    // "Registration successful. Please check your email to verify your account."
  } else {
    // "Registration failed. Please check your details and try again."
  }
}
\end{lstlisting}

On the backend:

\begin{itemize}[leftmargin=*,nosep]
  \item A user record is created.
  \item The password is stored as an Argon2 hash.
  \item A verification token is generated and stored as a hash.
  \item A mock email with a link like \texttt{https://example.com/verify-email?token=<token>} is logged.
\end{itemize}

\subsection{Verification Flow Example}

\begin{lstlisting}[language=Java]
import { verifyEmail } from '@/services/api';

async function handleVerify(): Promise<void> {
  const params = new URLSearchParams(window.location.search);
  const token = params.get('token');

  if (!token) {
    return;
  }

  const response = await verifyEmail(token);

  if (response.success) {
    // "Email verified successfully. You can now log in."
  } else {
    // "Verification failed or token has expired. Please request a new verification email."
  }
}
\end{lstlisting}

\subsection{Libraries and Tools}

\begin{itemize}[leftmargin=*,nosep]
  \item Axum -- Rust web framework
  \item SQLx -- Asynchronous SQL toolkit
  \item Argon2 -- Password hashing algorithm
  \item SHA2 -- Cryptographic hash functions
  \item \texttt{rand\_core} / \texttt{rand} -- Cryptographically secure RNG
  \item Protocol Buffers -- API schema and type generation
  \item Vue~3 -- Frontend framework
  \item Vuetify -- Material Design component library
\end{itemize}

\section{Version History}

\begin{longtable}{@{}lll@{}}
\toprule
\textbf{Version} & \textbf{Date} & \textbf{Changes} \\
\midrule
\endhead
1.0 & 2025-12-23 & Initial single-file registration module documentation \\
\bottomrule
\end{longtable}

\vspace{1em}

\noindent
\textbf{Document Generated}: 2025-12-23\\
\textbf{Last Updated}: 2025-12-23\\
\textbf{Status}: Complete\\
\textbf{Review Status}: Ready for review
